\section{Simulation}

The process of running the simulation for storyline creation as it does human mind is rather complex.
\begin{itemize}
 \item Objects/Subjects are being added in real time (upon demand).
 \item Objects are given their set of possible actions from a database when requested
 \item Each action/choice is created and checked by a complex mechanism of human mind
 \item Different parts are modeled with a different level of abstraction.
 \item There is a supervisor, capable of spotting the information, that is capable to amaze users on the overall result.
\end{itemize}

Generally speaking, now it is almost impossible to remove human from the storyline creation process. The framework is designed not to substitute the human but to relieve one from routine task and enhance some parts of the simulation. Here is the set of things, that computer can do better than human.
\begin{description}
 \item[Memory] There is always a way to run the computer simulation with slightly different parameters, as well as instantly save all the collected data on the drive, undamaged.
 \item[Logic] It is not easy for human mind to simulate processes , that involve a lot of math. Therefore such processes can be ignored, making the story a bit less realistic.
 \item[Background]There are a lot of routine actions usually going on but not attracting human attention therefore hard to keep track of. However this things create most of or surroundings and can sometimes play critical part in our stories if they happen in the right place in the right time.
 \item[Scale] Human mind can hardly work with big numbers.It is easy to imagine 2 humans fighting each other, but it is much harder to simulate 100 metal cubes falling down the slope.
\end{description}
\par
