\section{Storyline Creation}

All functions of a storyline developer can be specified using a\linebreak \(create\_storyline(W, T, B)\) function, where $W$ is a world view of the storyline developer, $T$ is the task the storyline developer was given, and B represents boundaries that are usually given by other developers in the team.
 The task of a storyline developer usually consists of statements.
 For example, a storyline developer might receive the request to create a narrative for an RPG (role-playing game) in medieval France with an army general as a main character. The output should be a detailed storyline, understandable for other members of the game development community.

Usually, a storyline consists of objects and characters descriptions, which are connected by actions that are performed by them or applied to them. Therefore, the storyline creation implies from storyline designer to execute several of the following actions:
\begin{itemize}
 \item interpretation of a given input to an operable form;
 \item addition of new elements to the story;
 \item simulation execution for event sequence determination;
 \item collection of the useful data (while ignoring the other one);
 \item interpretation of the simulation result to a shareable output.
\end{itemize}

A good storyline can often be described as the one that makes sense or is simply stunning. In order for the story to make sense the author should avoid inconsistencies between the game and the world view of ``judges''. The stunning storyline can be created with the knowledge about what triggers human emotions.
Both of this qualities are common for human but, unfortunately, almost impossible to be effectively recreated with any kind of intelligent systems. Intelligent systems can only efficiently perform the simulation itself.\par
Games can feature different types of storylines. In some cases storylines are linear, but most of the modern RPGs have tree-structured storylines. In other words, developers have to create a large number storylines with almost identical preconditions in order to provide a player the ability to influence the storyline with one's actions.
The problem of tree-structured storylines is that they are still look unnatural. One of the main reasons is because storyline developers (and game-designers in general) quickly become bored by describing the ``boring storyline branches''. For example, if the player in one of the modern RPGs decides to quit the main storyline and become a farmer, one most likely will not receive detailed quests such as ``grow the plants'' simply because the developers had no intention to waste their time on creation of the uncatchy part of the game. Even though such quests are simple and can be easily generated, lack of them makes games less realistic and more linear. 