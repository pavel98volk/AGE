\section{Software Overview}
Since multi-agent systems use to be more complex and are usually less efficient than, for example, systems based only on functional or OOP paradigm, agent oriented programming is not among the popular programming concepts. Nevertheless, due to being used in various scientific researches and AI based systems, agent notion was used in a multiple frameworks and libraries. The comparative table of them can be found in
\textbf{Appendix \ref{appendix:frameworks}}.

The problem with most frameworks is that they were originally designed to fulfill the certain set of tasks: and none of the tasks was similar to the one given in this paper.
Another problem is the commercial license. The most efficient systems have the highest prices. On the other hand there is an ability to completely customize open-source software.

Frameworks that are written to be used with custom or less popular languages are usually the least customizable and are useful only for simulating software, where productivity is of lesser concern than the ability to create the system rapidly.

The leading frameworks (in terms of the quantity of users) are usually implemented for Java or/and C++ programming languages that allows users to customize them in different ways.
