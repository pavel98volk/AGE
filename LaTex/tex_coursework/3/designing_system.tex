\section{Framework Design}
This section describes how to move from the abstract architecture created earlier to a more specific framework design. However, the design should not connected to any particular programming language.The results off this formalizing stage will be referenced as \textbf{Model1}
\begin{center}
    \textbf{Model1 specification}
\end{center}
At first there are more detailed specifications of internal structure of each class that was introduced in previous chapters.
\subsection{GameObject}
\texttt{GameObject} class is not given by default  any computational power. When event handler is executed, it uses computational power of the process that triggered this event handler. \texttt{GameObject} should not be active unless it was ``disturbed'' (the event handler was triggered). However, when ``disturbed'', \texttt{GameObject's} response is not bounded in time. For example, once activated, alarm might be sending alarm messages until it will be deactivated. To handle such ``disturbances'', \texttt{GameEnvironmentBroadcaster} class is created (one for each environment). The purpose of this class is to send specified message to each registered \texttt{GameObject} with specified time interval in between. Each \texttt{GameObject}  can send a message to the \texttt{GameEnvironment} with the desired return message and invocation interval to attempt registration within the \texttt{GameEnvironmentBroadcaster}. If the registration is successful, \texttt{GameObject} receives the corresponding message. The mechanism of deregistration is similar. In any moment of time \texttt{GameObject} can be also deregistered from the \texttt{GameEnvironmentBroadcaster} by the \texttt{GameEnvironment} itself. In this case it still receives a message with information about the deregistration event.\par
External and internal properties are saved as two map containers with name as a key to improve access speed. The environment can get, set, create and delete external parameters.
The map container for internal properties is useful in case \textit{Model0V} is being taken into account.
Events are passed as a message, response is given as a message as well.

\subsection{GameAgent}
One of the key differences between \texttt{GameObject} and \texttt{GameAgent} is that \texttt{GameAgent} can initiate communication with its environment. There are no stateless \texttt{GameAgents}. \texttt{GameAgent's} state should feature at least the assigned \texttt{GameEnvironment}.
\texttt{GameAgent} is should be implemented as autonomous entity, therefore the easy solution is to assign individual thread to each agent. Inefficiency of this solution is mostly seen when there is a huge amount of agents that can not influence the environment. Alternative solution is shown in the next subsection.

\subsection{GameEnvironment} \\
It is easier to specify \texttt{GameEnvironment} part by part.
\begin{itemize}
\item \textbf{GameEnvironmentMessageHandler} is used to analyze all messages that are passed through the environment. This peculiarity does make \break \texttt{GameEnvironmentMessageHandler} the main bottleneck of the system. To solve this problem, \texttt{GameEnvironmentMessageHandler} can use computational power of the sender agent when a message is passed. Also it should feature the individual computational power in order to maintain the environment when all other agents are deactivated.\par
    Along with sending messages, this part can invoke setters and getters of any entity in the corresponding \texttt{GameEnvironmentEntityContainer}.
\item \textbf{GameEnvironmentEntityContainer} should implement the effective mechanism of resource distribution among \texttt{GameAgents}. Since every time the message to be passed  \texttt{GameEnvironmentEntityContainer} $find$ function is invoked. That grants the \texttt{GameEnvironmentEntityContainer} for example, the ability to collect useful data about how many times the particular \texttt{GameObject} was invoked in the nearest period of time, store that data in the \texttt{GameObject's} external properties and use that data to calculate the amount of computational resources the agent will be given.

\item \textbf{GameEnvironmentResourcePool} is the module in which all resources as well as resource-giving classes are being held and given out. In particular, it contains \texttt{GameEnvironmentBroadcaster} object.
\end{itemize}

\subsection{GameEnvironmentObject/GameEnvironmentAgent} \\
The peculiarity of \texttt{GameDoubleObject} has additional $setInternalMessage$ function that can be invoked by its event handlers. To send a message to the external environment after being ``disturbed'' by the internal one or vice versa the \texttt{GameDoubleObject} should have the destination of the messages which it can send, therefore it should be given internal \texttt{GameEnvironmentMessageHandler} as well as the external one upon creation similarly to \texttt{GameAgent}.

\subsection{Advantages of Using \textit{Model0V} Voices in \textit{Model1}}\break
\paragraph{Attention driven resource distribution}
 To achieve better distribution of computational resources each agent can feature the external property of attention, assigned and constantly changed by the \texttt{GameObserverVoice} entity. If the attention to the entity drops below minimum, it is reasonable to exclude this entity from the equation by destroying or ``freezing'' it. 