\section{Examples of the Framework Usage}
The framework described above is meant for creating storylines for any possible game. That does not mean that its specification can not be extended to cover only the particular game genre or type. Let us use a RTS game such as Warcraft as an example.
The vague specification of this game can be found in \cite{warcraft}.
\begin{example}
Agents of the game can be divided into Commanders (such as the player), loyal units, loyal structures and neutral units and neutral structures.\par
The Commander's perceptions equals the combined perception of all loyal units and loyal structures under one's control. The external properties of a Commander feature resourc\breve{}es (gold and wood).\par
The loyal units and loyal structures are given the owner upon creation and are able to execute owner's requests as well as start interaction with their owner in any particular moment. For example, citadel can execute a Commander's request on creating a builder unit. When it receives gold, it tries to pass it to the owner of the citadel. Moreover, there is a possibility for loyal units / loyal structures to communicate between each other. These make coordinated behaviour between loyal units and even loyal structures possible.\par
 The neutral units have no owner, therefore must act on their own. \par
 All agents (except Commanders) can interact with all other type of agents that have a different owner, yet within some constraints. These constraints can be changed by relationship modifier between owners, for example the ``allies'' relationship.\par
  The objects in Warcraft can be divided in various subclasses, but each of them has its own position as its external property and is being contained by the map - the only environment in the game. There are agents that can contain some of the objects, but while doing so, they have no ability to create the simulation in which that objects would operate, so it will be simpler to depict the possession relationship as the external property of the owner.
 \end{example}
  The text description model of the game environment was given, but to create an interesting storyline it is essential to make the structure more complex, include dialogues or extensively describe the events. The first one is either made by creation of additional goals for units. The second require either collaboration with a human or an advanced AI technology. The third can be done by using GameObserverVoice class introduced in \textit{Model0V} architecture. Having all the information from this chapter on the table it is not hard to modify the system to the point where it would efficiently implement the recently created entities. Embedding some code from the actual game into the simulation environment can prove itself to be a good way to test whatever was embedded and develop the gameplay in general. 